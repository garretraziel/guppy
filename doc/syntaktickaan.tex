Syntaktická analýza je prováděna rekurzívním sestupem shora dolů. Vy\-tvo\-ři\-li jsme proto bezkontextovou gramatiku, pro kterou jsou terminálními symboly tokeny, které vrací lexikální analyzátor. Neterminální symboly popisují jednotlivé konstrukce jazyka IFJ11. Tuto gramatiku jsme vytýkáním a od\-stra\-ně\-ním levé rekurze převedli na LL-gramatiku, jejíž pravidla jsou na obrázku \ref{syn.prog}.

Počátečním neterminálním symbolem je symbol {\tt program}. Každý neterminální symbol je v analyzátoru reprezentován funkcí, která zpracovává pravou stranu pravidla. Pokud je na pravé straně pravidla terminální symbol, je kontrolováno, zda lexikální analyzátor právě takový token načetl. Pokud ne, jedná se o syntaktickou chybu. Neterminální symboly na pravé straně vedou k volání funkce, která tento symbol reprezentuje.

program $ \rightarrow $ definice-funkce   definice-funkcí   \textbf{;}   \textbf{EOF} \\
definice-funkce $ \rightarrow $	\textbf{function} \textbf{id} \textbf{(} parametry \textbf{)} deklarace příkazy \textbf{end} \\
definice-funkcí $ \rightarrow $ definice-funkce   definice-funkcí \\
definice-funkcí $ \rightarrow \varepsilon $ \\
parametry $ \rightarrow $	\textbf{identifier}   parametry-z \\
parametry $ \rightarrow \varepsilon $ \\
parametry-z $ \rightarrow $ \textbf{,}   \textbf{id} parametry-z \\
parametry-z $ \rightarrow 	\varepsilon $ \\
deklarace $ \rightarrow $	\textbf{local} \textbf{id} deklarace-z   deklarace \\
deklarace $ \rightarrow \varepsilon $ \\
deklarace-z $ \rightarrow $ \textbf{=} výraz \textbf{;} \\
deklarace-z $ \rightarrow $ \textbf{;} \\
literál $ \rightarrow $	\textbf{číslo} \\
literál $ \rightarrow $ \textbf{řetězec} \\
literál $ \rightarrow $ \textbf{nil} \\
literál $ \rightarrow $ \textbf{true} \\
literál $ \rightarrow $ \textbf{false} \\ 
sekvence-příkazů $ \rightarrow $	příkaz   \textbf{;}   sekvence-příkazů \\
sekvence-příkazů $ \rightarrow \varepsilon $ \\
příkaz $ \rightarrow $	\textbf{id} \textbf{=} assign-z \\
příkaz $ \rightarrow $	\textbf{write} \textbf{(} seznam-výrazů \textbf{)} \\
příkaz $ \rightarrow $	\textbf{if} výraz \textbf{then} sekvence-příkazů \textbf{else} sekvence-příkazů \textbf{end} \\
příkaz $ \rightarrow $	\textbf{while} výraz \textbf{do} sekvence-příkazů \textbf{end} \\
příkaz $ \rightarrow $	\textbf{return} výraz \\
příkaz $ \rightarrow $	\textbf{repeat} sekvence-příkazů \textbf{until} výraz \\
assign-z $ \rightarrow $ \textbf{read}   \textbf{(}  literál \textbf{)} \\
assign-z $ \rightarrow $ výraz \\
seznam-výrazů $ \rightarrow $ výraz seznam-výrazů-z \\
seznam-výrazů-z $ \rightarrow $ \textbf{,} výraz seznam-výrazů-z \\
seznam-výrazů-z $ \rightarrow \varepsilon $ \\


Výrazy jsou zpracovávány metodou zdola nahoru a je jim věnována následující podkapitola.
